
% approximate-inference.tex

\documentclass[dvipdfmx,notheorems,t]{beamer}

\usepackage{docmute}

% settings.tex

\AtBeginDvi{\special{pdf:tounicode 90ms-RKSJ-UCS2}}

\AtBeginSection[]{\frame[t]{\frametitle{目次}
	\tableofcontents[currentsection,hideallsubsections]}}

\AtBeginSubsection[]{\frame[t]{\frametitle{目次}
	\tableofcontents[currentsection,currentsubsection,subsectionstyle=show/shaded/hide]}}

\usefonttheme{professionalfonts}
\usetheme{Madrid}

\setbeamercovered{transparent=30} 
% \setbeamertemplate{navigation symbols}{}
\setbeamertemplate{frametitle}[default][left]
\setbeamertemplate{frametitle continuation}{}
\setbeamertemplate{enumerate items}[square]
\setbeamertemplate{caption}[numbered]

\let\oldframe\frame
\renewcommand\frame[1][allowdisplaybreaks,allowframebreaks,t]{\oldframe[#1]}

\usepackage{bxdpx-beamer}
\usepackage{pxjahyper}
\usepackage{minijs}

\usepackage{amsmath}
\usepackage{amssymb}
\usepackage{amsthm}
\usepackage{bm}

\DeclareMathOperator*{\argmax}{arg\,max}
\DeclareMathOperator*{\argmin}{arg\,min}
\DeclareMathOperator{\Tr}{Tr}
\DeclareMathOperator{\KL}{KL}
\DeclareMathOperator{\diag}{diag}

\usepackage[T1]{fontenc}
\usepackage[utf8]{inputenc}

\setbeamertemplate{theorems}[numbered]
\theoremstyle{definition}
\newtheorem{theorem}{定理}
\newtheorem{definition}{定義}
\newtheorem{proposition}{命題}
\newtheorem{lemma}{補題}
\newtheorem{corollary}{系}
\newtheorem{conjecture}{予想}
\newtheorem*{remark}{Remark}
\renewcommand{\proofname}{}

\renewcommand{\figurename}{図}
\renewcommand{\tablename}{表}

\renewcommand{\kanjifamilydefault}{\gtdefault}


\begin{document}

\section{近似推論法}

\subsection{変分推論}

\begin{frame}{変分推論}

\begin{itemize}
	\item 変分推論が必要だった理由
	\begin{itemize}
		\item 潜在変数に関する事後分布$p(\bm{Z} | \bm{X}, \bm{\theta})$の計算は、困難であることが多い
		\item どのような場合に困難になるのか、次の図\ref{fig:example-of-inference-problems}に示す
		\newline
		\item $p(\bm{Z} | \bm{X}, \bm{\theta})$の厳密な計算は諦める代わりに、\alert{別の確率分布で近似}したい
		\newline
		\item 別の確率分布で近似するとき、単純な項の積として表現できるといった、\alert{何らかの仮定を置く}
	\end{itemize}
\end{itemize}

\end{frame}

\begin{frame}{変分推論}

\begin{figure}[h]
	\centering
	\includegraphics[clip,scale=0.6,trim=2cm 17.8cm 2cm 5cm,page=647]{../goodfellow-deep-learning.pdf}
	\caption{事後分布$p(\bm{Z} | \bm{X})$の計算が困難な場合}
	\label{fig:example-of-inference-problems}
\end{figure}

\end{frame}

\begin{frame}{変分推論}

\begin{figure}[h]
	\centering
	\includegraphics[clip,scale=0.6,trim=2cm 10cm 2cm 11.8cm,page=647]{../goodfellow-deep-learning.pdf}
	\caption{事後分布$p(\bm{Z} | \bm{X})$の計算が困難な場合}
	\label{fig:example-of-inference-problems-2}
\end{figure}

\end{frame}

\begin{frame}{変分推論}

\begin{itemize}
	\item 事後分布$p(\bm{Z} | \bm{X})$の計算が困難な場合1
	\begin{itemize}
		\item グラフィカルモデルにおいて、\alert{潜在変数間の相互作用がある}場合、計算が困難になる
		\newline
		\item 左側の\alert{半制限付きボルツマンマシン}では、全ての潜在変数の組み合わせ間で、接続がある
		\item 従って、潜在変数間に\alert{依存関係}が存在し、事後分布の計算が手に負えない
		\newline
		\item 白丸で描かれた潜在変数を$\bm{Z} = \left\{ z_1, z_2, z_3 \right\}$、灰色で描かれた観測データを$\bm{X}$とおくと、事後分布$p(\bm{Z} | \bm{X})$は、例えば次のようになる
		\begin{eqnarray}
			&& p(z_1, z_2, z_3 | \bm{X}) \nonumber \\
			&=& p(z_1 | \bm{X}, z_2, z_3) p(z_2 | \bm{X}, z_3) p(z_3 | \bm{X})
		\end{eqnarray}
	\end{itemize} \
	
	\item 事後分布$p(\bm{Z} | \bm{X})$の計算が困難な場合2
	\begin{itemize}
		\item 中央は、層間の結合がない潜在変数の層で構成される、\alert{深層ボルツマンマシン}を表す
		\item 潜在変数の層間の結合があるため、事後分布の計算が手に負えない
		\newline
		\item 上の層の潜在変数を$\bm{Z}_1 = \left\{ z_{11}, z_{12}, z_{13} \right\}$、中間層の潜在変数を$\bm{Z}_2 = \left\{ z_{21}, z_{22}, z_{23} \right\}$、灰色で描かれた観測データを$\bm{X}$とおくと、事後分布は、例えば次のようになる
		\begin{eqnarray}
			&& p(\bm{Z}_1, \bm{Z}_2 | \bm{X}) \nonumber \\
			&=& p(\bm{Z}_2 | \bm{X}) p(\bm{Z}_1 | \bm{Z}_2) \\
			&=& p(z_{21}, z_{22}, z_{23} | \bm{X}) p(z_{11}, z_{12}, z_{13} | \bm{Z}_2) \nonumber \\
			&=& p(z_{21} | \bm{X}) p(z_{22} | \bm{X}) p(z_{23} | \bm{X}) \nonumber \\
			&& \qquad p(z_{11} | \bm{Z}_2) p(z_{12} | \bm{Z}_2) p(z_{13} | \bm{Z}_2)
		\end{eqnarray}
	\end{itemize}
\end{itemize}

\end{frame}

\subsection{座標降下法}

\begin{frame}{変分推論}

\begin{itemize}
	\item 変分推論の目的
	\begin{itemize}
		\item 同時分布$p(\bm{X}, \bm{Z})$が求まっているときに、\alert{事後分布}$p(\bm{Z} | \bm{X})$と、\alert{エビデンス}$p(\bm{X})$の近似を求める
	\end{itemize} \
	
	\item 注意点
	\begin{itemize}
		\item 観測変数と潜在変数をまとめて、$\bm{X}, \bm{Z}$とおく
		\item $p(\bm{X})$は、確率モデルからデータ$\bm{X}$が生起する確率である
		\item \alert{データからみたモデルの好み}と解釈できるから、$p(\bm{X})$を\alert{モデルエビデンス}という
	\end{itemize}
\end{itemize}

\end{frame}

\begin{frame}{変分推論}

\begin{itemize}
	\item 周辺分布の対数$\ln p(\bm{X})$の分解
	\begin{itemize}
		\item EMアルゴリズムのときと同様であり、次のように分解できる
		\begin{equation}
			\ln p(\bm{X}) = \mathcal{L}(q) + \KL (q || p)
		\end{equation}
		
		\item 但し、$\mathcal{L}(q)$と$\KL (q || p)$は次のように定義した
		\begin{eqnarray}
			\mathcal{L}(q) &=& \int q(\bm{Z}) \ln \frac{p(\bm{X}, \bm{Z})}{q(\bm{Z})} d\bm{Z} \\
			\KL (q || p) &=& -\int q(\bm{Z}) \ln \frac{p(\bm{Z} | \bm{X})}{q(\bm{Z})} d\bm{Z}
		\end{eqnarray}
	\end{itemize} \
		
	\item パラメータ$\bm{\theta}$の扱い
	\begin{itemize}
		\item EMアルゴリズムとは異なり、\color{red}パラメータ$\bm{\theta}$がどこにも現れていない\normalcolor
		% \item ここでは、全てのパラメータに事前分布が与えられた、\alert{完全にベイズ的なモデル}を考えている
		% \item パラメータも、確率変数として$\bm{Z}$の中に含まれている
		\item \alert{パラメータも潜在変数として扱っている}ので、パラメータベクトルは明示的には書かない
		\item ここでは、パラメータ$\bm{\theta}$については、あまり気にしない
		\newline
		\item ここでは連続潜在変数について考えるが、離散潜在変数であれば、積分を$\bm{Z}$に関する総和に置き換えればよい
	\end{itemize} \
	
	\item 下界$\mathcal{L}(q)$を最適化する動機
	\begin{itemize}
		\item $\mathcal{L}(q)$はエビデンスの対数$\ln p(\bm{X})$の下界であるから、\alert{エビデンス下界}(Evidence lower bound, \alert{ELBO})ともいう
		\item または、負の\alert{変分自由エネルギー}(Variational free energy)という
		\newline
		
		\item $\ln p(\bm{X})$は$q$には依存しないため、定数項とみなせる
		\item 従って、$\mathcal{L}(q)$を$q$について最大化することは、$\KL (q || p)$の最小化に相当
		\item このとき、分布$q(\bm{Z})$を真の事後分布$p(\bm{Z} | \bm{X})$に近づけられる
		\item $q(\bm{Z}) = p(\bm{Z} | \bm{X})$が分かれば、データ$\bm{X}$から、\color{red}潜在変数やパラメータ$\bm{Z}$が得られる\normalcolor
	\end{itemize} \
	
	\framebreak
	
	\item 下界$\mathcal{L}(q)$の最適化
	\begin{itemize}
		\item EMアルゴリズムのときと同じように、下界$\mathcal{L}(q)$を、分布$q(\bm{Z})$について最大化する
		\item これは、KLダイバージェンス$\KL (q || p)$を最小化することと等価である
		\newline
		\item 従って、\color{red}もし$q(\bm{Z})$を任意の分布にしてよければ\normalcolor 、$q(\bm{Z}) = p(\bm{Z} | \bm{X})$とおいて、KLダイバージェンスを$0$にすればよい
		\newline
		\item しかしここでは、\color{red}真の事後分布$p(\bm{Z} | \bm{X})$を求めることは不可能\normalcolor と仮定する
	\end{itemize} \
	
	\item 分布$q(\bm{Z})$の近似
	\begin{itemize}
		\item 計算コストを削減するために、$q(\bm{Z})$の形をある程度\alert{制限する}
		\item 制限したクラスの$q(\bm{Z})$の中で、KLダイバージェンス$\KL (q || p)$を最小化するものを探す
	\end{itemize} \
	
	\item 変分推論の目的
	\begin{itemize}
		\item 分布のクラスを制限することで、$q(\bm{Z})$を計算可能にすること
		\item 表現力が豊かなクラスを使うことで、真の事後分布$p(\bm{Z} | \bm{X})$を良く近似する
		\newline
		\item 計算可能な分布のクラスの中で、\alert{可能な限り豊かな表現力を持つ}ものを選びたい
		\item 表現力が豊かな分布を使うことは、真の事後分布を、精度良く近似することにつながるのであって、従って\alert{過学習は発生しない}
	\end{itemize}
\end{itemize}

\end{frame}

\begin{frame}{変分推論}

\begin{itemize}
	\item 分布$q(\bm{Z})$のクラスを制限する方法
	\begin{itemize}
		\item 例えば分布$q(\bm{Z})$を、\alert{パラメトリックな分布に限定}することができる
		\item 即ち、パラメータベクトル$\bm{\omega}$によって$q(\bm{Z} | \bm{\omega})$と記述されるような、分布に制限する
		\newline
		\item 分布$q(\bm{Z})$を、ガウス分布などの、何らかの特別なパラメトリックな分布と仮定することに相当
	\end{itemize}
\end{itemize}

\end{frame}

\begin{frame}{変分推論}

\begin{itemize}
	\item クラスを制限する別の方法(\alert{平均場近似})
	\begin{itemize}
		\item 分布$q(\bm{Z})$のクラスを制限する別の方法として、\alert{平均場近似}がある
		\newline
		\item 潜在変数$\bm{Z}$を、$M$個の\color{red}互いに排反なグループ$\left\{ \bm{Z}_1, \ldots, \bm{Z}_M \right\}$に分割\normalcolor
		\item 分布$q(\bm{Z})$が、\alert{これらのグループによって分解されると仮定}
		\begin{equation}
			q(\bm{Z}) = \prod_{i = 1}^M q_i(\bm{Z}_i)
		\end{equation}
		
		\item 分布$q$について、\alert{これ以上の仮定はしない}
		\item 従って、各因子$q_i(\bm{Z}_i)$の関数形については、\alert{何の制限も課さない}
		\newline
		\item 平均場近似とは、元々は物理学における用語である
	\end{itemize}
\end{itemize}

\end{frame}

\begin{frame}{変分推論}

\begin{itemize}
	\item 下界$\mathcal{L}(q)$の最大化
	\begin{itemize}
		\item $q(\bm{Z}) = \prod_i q_i(\bm{Z}_i)$と分解できるような分布$q(\bm{Z})$の中で、\color{red}下界$\mathcal{L}(q)$を最大にするもの\normalcolor を探す
		\newline
		\item $\mathcal{L}(q)$を$q(\bm{Z})$について最大化するために、$\mathcal{L}(q)$を各因子$q_i(\bm{Z}_i)$について\color{red}順番に最大化\normalcolor していく
		\item $\mathcal{L}(q)$に$q(\bm{Z}) = \prod_i q_i(\bm{Z}_i)$を代入して、因子の一つ$q_j(\bm{Z}_j)$に関する\alert{依存項}を抜き出してみよう
		\begin{eqnarray}
			\mathcal{L}(q) &=& \int q(\bm{Z}) \ln \frac{p(\bm{X}, \bm{Z})}{q(\bm{Z})} d\bm{Z} \\
			&=& \int \left( \prod_i q_i(\bm{Z}_i) \right) \ln \frac{p(\bm{X}, \bm{Z})}{q(\bm{Z})} d\bm{Z} \\
			&=& \int \prod_i q_i(\bm{Z}_i) \left( \ln p(\bm{X}, \bm{Z}) - \sum_i \ln q_i(\bm{Z}_i) \right) d\bm{Z} \\
			&=& \int \prod_i q_i(\bm{Z}_i) \left( \ln p(\bm{X}, \bm{Z}) \right) d\bm{Z} - \nonumber \\
			&& \qquad \int \prod_i q_i(\bm{Z}_i) \left( \sum_i \ln q_i(\bm{Z}_i) \right) d\bm{Z}
		\end{eqnarray}
		ここで第1項は
		\begin{eqnarray}
			&& \int \prod_i q_i(\bm{Z}_i) \left( \ln p(\bm{X}, \bm{Z}) \right) d\bm{Z} \nonumber \\
			&=& \int q_j(\bm{Z}_j) \left( \prod_{i \neq j} q_i(\bm{Z}_i) \right) \left( \ln p(\bm{X}, \bm{Z}) \right) d\bm{Z}
		\end{eqnarray}
		$d\bm{Z} = d\bm{Z}_1 d\bm{Z}_2 \cdots d\bm{Z}_M$であるから
		\begin{eqnarray}
			&=& \int q_j(\bm{Z}_j) \left( \prod_{i \neq j} q_i(\bm{Z}_i) \right) \left( \ln p(\bm{X}, \bm{Z}) \right) d\bm{Z}_1 d\bm{Z}_2 \cdots d\bm{Z}_M \\
			&=& \int q_j(\bm{Z}_j) \left( \ln p(\bm{X}, \bm{Z}) \right) \left( \prod_{i \neq j} q_i(\bm{Z}_i) \right) \left( \prod_{i \neq j} d\bm{Z}_i \right) d\bm{Z}_j \\
			&=& \int q_j(\bm{Z}_j) \left( \int \left( \ln p(\bm{X}, \bm{Z}) \right) \prod_{i \neq j} q_i(\bm{Z}_i) d\bm{Z}_i \right) d\bm{Z}_j \\
			&=& \int q_j(\bm{Z}_j) \ln \widetilde{p}(\bm{X}, \bm{Z}_j) d\bm{Z}_j
		\end{eqnarray}
		
		\item 但し、新しい分布$\widetilde{p}(\bm{X}, \bm{Z}_j)$は以下の式で定義した(積分の結果であるため、定数項が出現する)
		\begin{equation}
			\ln \widetilde{p}(\bm{X}, \bm{Z}_j) = \mathbb{E}_{i \neq j}[\ln p(\bm{X}, \bm{Z})] + \mathrm{Const.}
		\end{equation}
		
		\item 記法$\mathbb{E}_{i \neq j}$は、$i \neq j$をみたす全ての分布$q_i(\bm{Z}_i)$による、期待値を取ることを表す
		\begin{equation}
			\mathbb{E}_{i \neq j}[\ln p(\bm{X}, \bm{Z})] = \int \left( \ln p(\bm{X}, \bm{Z}) \right) \prod_{i \neq j} q_i(\bm{Z}_i) d\bm{Z}_i
		\end{equation}
		
		\item 第2項は
		\begin{eqnarray}
			&& \int \prod_i q_i(\bm{Z}_i) \left( \sum_i \ln q_i(\bm{Z}_i) \right) d\bm{Z} \nonumber \\
			&=& \sum_i \int \prod_i q_i(\bm{Z}_i) \left( \ln q_i(\bm{Z}_i) \right) d\bm{Z} \\
			&=& \sum_i \int \prod_k q_k(\bm{Z}_k) \left( \ln q_i(\bm{Z}_i) \right) d\bm{Z}_1 d\bm{Z}_2 \cdots d\bm{Z}_M \\
			&=& \int \prod_k q_k(\bm{Z}_k) \left( \ln q_j(\bm{Z}_j) \right) d\bm{Z}_1 d\bm{Z}_2 \cdots d\bm{Z}_M + \nonumber \\
			&& \qquad \sum_{i \neq j} \int \prod_k q_k(\bm{Z}_k) \left( \ln q_i(\bm{Z}_i) \right) d\bm{Z}_1 d\bm{Z}_2 \cdots d\bm{Z}_M
		\end{eqnarray}
		但し
		\begin{eqnarray}
			&& \int \prod_k q_k(\bm{Z}_k) \left( \ln q_j(\bm{Z}_j) \right) d\bm{Z}_1 d\bm{Z}_2 \cdots d\bm{Z}_M \\
			&=& \int q_j(\bm{Z}_j) \ln q_j(\bm{Z}_j) \left( \prod_{k \neq j} q_k(\bm{Z}_k) \right) \left( \prod_{k \neq j} d\bm{Z}_k \right) d\bm{Z}_j \\
			&=& \int q_j(\bm{Z}_j) \ln q_j(\bm{Z}_j) \left( \int \prod_{k \neq j} q_k(\bm{Z}_k) d\bm{Z}_k \right) d\bm{Z}_j \\
			&=& \int q_j(\bm{Z}_j) \ln q_j(\bm{Z}_j) \prod_{k \neq j} \underbrace{\left( \int q_k(\bm{Z}_k) d\bm{Z}_k \right)}_{=1} d\bm{Z}_j \\
			&=& \int q_j(\bm{Z}_j) \ln q_j(\bm{Z}_j) d\bm{Z}_j
		\end{eqnarray}
		であるほか
		\begin{eqnarray}
			&& \sum_{i \neq j} \int \prod_k q_k(\bm{Z}_k) \left( \ln q_i(\bm{Z}_i) \right) d\bm{Z}_1 d\bm{Z}_2 \cdots d\bm{Z}_M \nonumber \\
			&=& \sum_{i \neq j} \int q_i(\bm{Z}_i) q_j(\bm{Z}_j) \left( \prod_{k \neq i, j} \ln q_k(\bm{Z}_k) \right) \nonumber \\
			&& \qquad \left( \ln q_i(\bm{Z}_i) \right) \left( \prod_{k \neq i, j} d\bm{Z}_k \right) d\bm{Z}_i d\bm{Z}_j \\
			&=& \sum_{i \neq j} \underbrace{\left( \int q_j(\bm{Z}_j) d\bm{Z}_j \right)}_{=1} \left( \int \prod_{k \neq i, j} \ln q_k(\bm{Z}_k) d\bm{Z}_k \right) \nonumber \\
			&& \qquad \int \ln q_i(\bm{Z}_i) q_i(\bm{Z}_i) d\bm{Z}_i \\
			&=& \sum_{i \neq j} \prod_{k \neq i, j} \underbrace{\left( \int \ln q_k(\bm{Z}_k) d\bm{Z}_k \right)}_{=\mathrm{Const.}} \underbrace{\int \ln q_i(\bm{Z}_i) q_i(\bm{Z}_i) d\bm{Z}_i}_{=\mathrm{Const.}} \\
			&=& \sum_{i \neq j} \mathrm{Const.} = \mathrm{Const.}
		\end{eqnarray}
		となるから結局
		\begin{eqnarray}
			&& \int \prod_i q_i(\bm{Z}_i) \left( \sum_i \ln q_i(\bm{Z}_i) \right) d\bm{Z} \nonumber \\
			&=& \int q_j(\bm{Z}_j) \ln q_j(\bm{Z}_j) d\bm{Z}_j + \mathrm{Const.}
		\end{eqnarray}
		
		\item 従って、下界$\mathcal{L}(q)$から$q_j(\bm{Z}_j)$に依存する項を取り出すと
		\begin{eqnarray}
			&& \mathcal{L}(q) = \int q_j(\bm{Z}_j) \ln \widetilde{p}(\bm{X}, \bm{Z}_j) d\bm{Z}_j - \nonumber \\
			&& \qquad \int q_j(\bm{Z}_j) \ln q_j(\bm{Z}_j) d\bm{Z}_j + \mathrm{Const.}
		\end{eqnarray}
		但し
		\begin{equation}
			\ln \widetilde{p}(\bm{X}, \bm{Z}_j) = \mathbb{E}_{i \neq j}[\ln p(\bm{X}, \bm{Z})] = \int \ln p(\bm{X}, \bm{Z}) \prod_{i \neq j} q_i(\bm{Z}_i) d\bm{Z}_i
		\end{equation}
		
		\item $\mathcal{L}(q)$を、$i \neq j$である全ての$q_i(\bm{Z}_i)$について\alert{固定}した上で、$q_j(\bm{Z}_j)$について最大化することになる
		\item $q_j(\bm{Z}_j)$について可能な全ての分布の中で、$\mathcal{L}(q)$を最大にするようなものを選ぶ
		
		\item $\mathcal{L}(q)$は次のように変形できる
		\begin{eqnarray}
			\mathcal{L}(q) &=& \int q_j(\bm{Z}_j) \ln \frac{\widetilde{p}(\bm{X}, \bm{Z}_j)}{q_j(\bm{Z}_j)} d\bm{Z}_j + \mathrm{Const.} \\
			&=& - \KL \left( q_j(\bm{Z}_j) || \widetilde{p}(\bm{X}, \bm{Z}_j) \right) + \mathrm{Const.}
		\end{eqnarray}
		
		\item これより、$\mathcal{L}(q)$は$q_j(\bm{Z}_j)$と$\widetilde{p}(\bm{X}, \bm{Z}_j)$の間の、負のKLダイバージェンスとなっている
		\item そして、$\mathcal{L}(q)$の$q_j(\bm{Z}_j)$に関する最大化は、\alert{KLダイバージェンスの最小化}と等価
		\newline
		\item KLダイバージェンスを最小にするためには、$q_j(\bm{Z}_j) = \widetilde{p}(\bm{X}, \bm{Z}_j)$とすればよい
		\newline
		\item 従って、$q_j(\bm{Z}_j)$の最適解は、次のように書ける
		\begin{equation}
			\color{red}\ln q_j^*(\bm{Z}_j) = \mathbb{E}_{i \neq j}[\ln p(\bm{X}, \bm{Z})] + \mathrm{Const.}\normalcolor
		\end{equation}
	\end{itemize}
\end{itemize}

\end{frame}

\begin{frame}{変分推論}

\begin{itemize}
	\item 下界$\mathcal{L}(q)$を最大化する$\ln q_j(\bm{Z}_j)$の解
	\begin{itemize}
		\item $q_j(\bm{Z}_j)$の最適解は次のように書けた
		\begin{equation}
			\ln q_j^*(\bm{Z}_j) = \mathbb{E}_{i \neq j}[\ln p(\bm{X}, \bm{Z})] + \mathrm{Const.}
		\end{equation}
		
		\item 上式は次のことを意味している
		\item 因子$q_j(\bm{Z}_j)$の最適解の対数$\ln q_j^*(\bm{Z}_j)$は、観測データ$\bm{X}$と潜在変数$\bm{Z}$の\alert{同時分布の対数}$\ln p(\bm{X}, \bm{Z})$\alert{を考え}、$i \neq j$である他の因子$q_i(\bm{Z}_i)$について\alert{期待値を取ったもの}である
		\newline
		
		\item 定数項は、正規化することで得られるので、結局次のようになる
		\begin{equation}
			q_j^*(\bm{Z}_j) = \frac{\exp(\mathbb{E}_{i \neq j}[\ln p(\bm{X}, \bm{Z})])}{\displaystyle \int \exp(\mathbb{E}_{i \neq j}[\ln p(\bm{X}, \bm{Z})]) d\bm{Z}_j}
		\end{equation}
		
		\item 正規化定数は必要に応じて計算すればよいので、取り敢えず無視できる
		\newline
		
		\item 最適解の式は、$q(\bm{Z})$の分解の数だけ得られるので、$\left\{ q_i(\bm{Z}_i) \right\}$に関する$M$本の連立方程式となる
		\item この方程式は、分布$q(\bm{Z})$が$M$個の因子に分解されるという仮定の下で、下界$\mathcal{L}(q)$の最大値が満たすべき条件である
		\newline
		
		\item $\ln q_j^*(\bm{Z}_j)$の右辺は、$i \neq j$である$q_i(\bm{Z}_i)$の期待値に依存するため、$q_j^*(\bm{Z}_j)$を陽に求めることができない
		\item そこで、下界$\mathcal{L}(q)$は次のように最適化される(\alert{重要})
		\newline
		\item $i \neq j$である全ての$q_i(\bm{Z}_i)$を\alert{固定}した状態で、$q_j(\bm{Z}_j)$を最適化することを、全ての$j = 1, \ldots, M$について繰り返す手続きを、\alert{座標降下法}という
	\end{itemize}
\end{itemize}

\end{frame}

\begin{frame}{変分推論}

\begin{itemize}
	\item 下界$\mathcal{L}(q)$の最適化 (\alert{座標降下法})
	\begin{enumerate}
		\item 全ての因子$q_j(\bm{Z}_j)$を適当に初期化する
		\newline
		\item 各因子を、以下の式を使って更新する \label{enum:variational-inference-step}
		\begin{eqnarray}
			&& \ln q_j(\bm{Z}_j) = \mathbb{E}_{i \neq j}[\ln p(\bm{X}, \bm{Z})] + \mathrm{Const.} \\
			&& \mathbb{E}_{i \neq j}[\ln p(\bm{X}, \bm{Z})] = \int \ln p(\bm{X}, \bm{Z}) \prod_{i \neq j} q_i(\bm{Z}_i) d\bm{Z}_i
		\end{eqnarray}
		即ち、因子$q_j(\bm{Z}_j)$を、他の全ての因子の現在の値$q_i(\bm{Z}_i)$を使って改良する
		\newline
		\item (\ref{enum:variational-inference-step})を、下界$\mathcal{L}(q)$が収束するまで繰り返す
	\end{enumerate}
\end{itemize}

\end{frame}

\begin{frame}{変分推論}

\begin{itemize}
	\item ここまでの話の流れ
	\begin{enumerate}
		\item $\ln p(\bm{X}) = \mathcal{L}(q) + \KL (q(\bm{Z}) || p(\bm{Z} | \bm{X}))$であるから、\alert{エビデンス下界}$\mathcal{L}(q)$を\color{red}$q$について最大化\normalcolor すれば、$\KL (q || p) = 0$とでき、従って$q(\bm{Z}) = p(\bm{Z} | \bm{X})$を得る
		\newline
		\item $q(\bm{Z}) = p(\bm{Z} | \bm{X})$が分かれば、データ$\bm{X}$から、\color{red}潜在変数やパラメータ$\bm{Z}$が得られる\normalcolor (パラメータは潜在変数$\bm{Z}$に含まれている)
		\newline
		\item しかし、事後分布$q(\bm{Z}) = p(\bm{Z} | \bm{X})$は\alert{計算不可能}なので、何らかの方法で\alert{近似}するしかない
		\newline
		\item 近似するといっても、計算可能でなければならないので、$q(\bm{Z})$の形には、通常\alert{何らかの制限}を課す
		\newline
		\item $q(\bm{Z})$を、\alert{パラメトリックな分布}$q(\bm{Z} | \bm{\omega})$と仮定することがある
		\newline
		\item または、$q(\bm{Z})$を、$\prod_i q_i(\bm{Z}_i)$のように分解できるとする(\alert{平均場近似})
		\newline
		\item \alert{平均場近似}を行うとき、各因子$q_j(\bm{Z}_j)$の最適解は、$\ln q_j^*(\bm{Z}_j) = \mathbb{E}_{i \neq j}[\ln p(\bm{X}, \bm{Z})] + \mathrm{Const.}$であった
		\newline
		\item 全ての因子$\left\{ q_j(\bm{Z}_j) \right\}$を\alert{同時に最適化することはできない}
		\newline
		\item 下界$\mathcal{L}(q)$を、各因子$q_j(\bm{Z}_j)$について\alert{順番に最適化}することはできる
	\end{enumerate} \
	
	\item これからの話の流れ
	\begin{itemize}
		\item MAP推定と最尤推定は、変分推論の\alert{特殊な場合}であることを確認する
	\end{itemize}
\end{itemize}

\end{frame}

\subsection{MAP推定および最尤推定と変分推論}

\begin{frame}{変分推論}

\begin{itemize}
	\item MAP推定および最尤推定の変分推論からの導出
	\begin{itemize}
		\item 変分推論で得たいのは、潜在変数に関する事後分布$p(\bm{Z} | \bm{X})$である
		\item この変分推論の\alert{特殊ケース}が、MAP推定や最尤推定であることを導く
		\newline
		\item 変分推論では、エビデンス下界$\mathcal{L}(q)$を$q$について最大化し、従って\color{red}$\KL(q || p)$を最小化\normalcolor する
		\item いま、分布$q(\bm{Z})$が、次の\alert{デルタ関数}であるとする
		\item 特定の値$\bm{Z} = \widehat{\bm{Z}}$についてのみ確率が非零になる、無限に鋭い分布
		\begin{equation}
			q(\bm{Z}) = \delta(\bm{Z} - \widehat{\bm{Z}})
		\end{equation}
		
		\item このとき、KLダイバージェンス$\KL(q || p)$は次のようになる
		\begin{eqnarray}
			&& \KL(q || p) \nonumber \\
			&\equiv& \KL(q(\bm{Z}) || p(\bm{Z} | \bm{X})) \nonumber \\
			&=& - \int q(\bm{Z}) \ln \frac{p(\bm{Z} | \bm{X})}{q(\bm{Z})} d\bm{Z} \nonumber \\
			&=& - \int q(\bm{Z}) \ln p(\bm{Z} | \bm{X}) d\bm{Z} + \int q(\bm{Z}) \ln q(\bm{Z}) d\bm{Z} \\
			&=& - \int \delta(\bm{Z} - \widehat{\bm{Z}}) \ln p(\bm{Z} | \bm{X}) d\bm{Z} + \nonumber \\
			&& \qquad \int \delta(\bm{Z} - \widehat{\bm{Z}}) \ln \delta(\bm{Z} - \widehat{\bm{Z}}) d\bm{Z} \\
			&=& - \ln p(\widehat{\bm{Z}} | \bm{X}) + \ln \delta(\widehat{\bm{Z}} - \widehat{\bm{Z}}) \\
			&=& - \ln p(\widehat{\bm{Z}} | \bm{X}) + \mathrm{Const.}
		\end{eqnarray}
		
		\item これを$q(\bm{Z})$について最小化することは、$\widehat{\bm{Z}}$について最小化することに相当する
		\item よって$\widehat{\bm{Z}}$の最適解$\widehat{\bm{Z}}^*$は
		\begin{eqnarray}
			\widehat{\bm{Z}}^* &=& \argmin_{\widehat{\bm{Z}}} \left( -\ln p(\widehat{\bm{Z}} | \bm{X}) \right) \\
			&=& \argmax_{\widehat{\bm{Z}}} \ln p(\widehat{\bm{Z}} | \bm{X}) \\
			&=& \argmax_{\widehat{\bm{Z}}} p(\widehat{\bm{Z}} | \bm{X}) \\
			&=& \argmax_{\widehat{\bm{Z}}} p(\bm{X} | \widehat{\bm{Z}}) p(\widehat{\bm{Z}})
		\end{eqnarray}
		となるから、\alert{MAP推定}(\alert{最大事後確率}推定)の式と一致する
		
		\item 更に、$p(\widehat{\bm{Z}}) = \mathrm{Const.}$、即ち$\widehat{\bm{Z}}$に関する事前の情報がないとすると
		\begin{eqnarray}
			\widehat{\bm{Z}}^* &=& \argmax_{\widehat{\bm{Z}}} p(\bm{X} | \widehat{\bm{Z}}) p(\widehat{\bm{Z}}) \nonumber \\
			&=& \argmax_{\widehat{\bm{Z}}} p(\bm{X} | \widehat{\bm{Z}})
		\end{eqnarray}
		となるから、これは\alert{最尤推定}の式に一致する
	\end{itemize}
\end{itemize}

\end{frame}

\subsection{変分近似による弊害}

\begin{frame}{変分推論}

\begin{itemize}
	\item これまでの話の流れ
	\begin{enumerate}
		\item 変分推論から、MAP推定と最尤推定が導出できることを確認した
		\newline
		\item MAP推定は、$q(\bm{Z})$を無限に鋭い確率分布(デルタ関数)として、$\bm{Z}$が特定の値しか取らないと仮定した場合であった
		\item 最尤推定は、MAP推定において、$\bm{Z}$の事前確率分布を設けない場合であった
	\end{enumerate} \
	
	\item これからの話の流れ
	\begin{itemize}
		\item 話題を変えて、分解$q(\bm{Z}) = \prod_i q_i(\bm{Z}_i)$によって$p(\bm{Z} | \bm{X})$を近似するときの、弊害を調べる
	\end{itemize}
\end{itemize}

\end{frame}

\begin{frame}{変分推論}

\begin{itemize}
	\item $q(\bm{Z})$の分解による近似の性質
	\begin{itemize}
		\item 変分推論では、真の事後分布$q(\bm{Z}) = p(\bm{Z} | \bm{X})$を、分解により近似する
		\item 分解で近似することによって、\alert{どのような不正確さが生じるのか?}
	\end{itemize} \
	
	\item ガウス分布の分解による近似
	\begin{itemize}
		\item ガウス分布を、\alert{分解されたガウス分布}で近似することを考えてみよう
		\item 分解による近似で、どのような問題が起こるのか考えてみよう
		\newline
		\item 2つの変数$\bm{z} = (z_1, z_2)$間には、\alert{相関がある}とする
		\item $\bm{z}$はガウス分布$p(\bm{z}) = \mathcal{N}(\bm{z} | \bm{\mu}, \bm{\Lambda}^{-1})$に従っているとする
		\begin{equation}
			\bm{\mu} = \left[ \begin{array}{l} \mu_1 \\ \mu_2 \end{array} \right], \qquad \bm{\Lambda} = \left[ \begin{array}{ll} \Lambda_{11} & \Lambda_{12} \\ \Lambda_{21} & \Lambda_{22} \end{array} \right]
		\end{equation}
		
		\item 精度行列$\bm{\Lambda}$は対称行列であるから、$\Lambda_{12} = \Lambda_{21}$
		\newline
		\item この分布$p(\bm{z})$を、分解されたガウス分布$q(\bm{z}) = q_1(z_1) q_2(z_2)$で近似する
		\item 各因子$q_1(z_1), q_2(z_2)$の関数形については\alert{何の仮定も置いていない}ことに注意
	\end{itemize} \
	
	\item 最適な因子$q_1(z_1), q_2(z_2)$の計算
	\begin{itemize}
		\item 最適な因子$q_1^*(z_1)$を、先程の結果を使って求める
		\begin{equation}
			\ln q_j(\bm{Z}_j) = \mathbb{E}_{i \neq j}[\ln p(\bm{X}, \bm{Z})] + \mathrm{Const.}
		\end{equation}
		
		\item 従って、$q_1^*(z_1)$を計算する式は次のようになる
		\begin{eqnarray}
			&& \ln q_1^*(z_1) = \mathbb{E}_{z_2}[\ln p(\bm{z})] + \mathrm{Const.} \\
			&& \mathbb{E}_{z_2}[\ln p(\bm{z})] = \int \ln p(\bm{z}) q_2(z_2) dz_2
		\end{eqnarray}
		
		\item 上式の右辺では、$z_1$に依存する項だけを考えればよい
		\item $z_1$の関数を求めようとしているため
		\item $z_1$に依存しない項は、全て定数項(正規化定数)に含まれてしまうため
		\newline
		\item 従って$q_1^*(z_1)$は
		\begin{eqnarray}
			\ln q_1^*(z_1) &=& \mathbb{E}_{z_2}[\ln p(\bm{z})] + \mathrm{Const.} \\
			&=& \mathbb{E}_{z_2} \left[ \ln \mathcal{N}(\bm{z} | \bm{\mu}, \bm{\Lambda}^{-1}) \right] + \mathrm{Const.}
		\end{eqnarray}
		但し
		\begin{eqnarray}
			&& \ln \mathcal{N}(\bm{z} | \bm{\mu}, \bm{\Lambda}^{-1}) \nonumber \\
			&=& \ln \left( \frac{1}{(2\pi)^\frac{2}{2}} \frac{1}{|\bm{\Lambda}^{-1}|^\frac{1}{2}} \exp \left( -\frac{1}{2} (\bm{z} - \bm{\mu})^T \left( \bm{\Lambda}^{-1} \right)^{-1} (\bm{z} - \bm{\mu}) \right) \right) \nonumber \\
			&=& \ln \left( \frac{1}{2\pi} \frac{1}{|\bm{\Lambda}|^{-\frac{1}{2}}} \exp \left( -\frac{1}{2} (\bm{z} - \bm{\mu})^T \bm{\Lambda} (\bm{z} - \bm{\mu}) \right) \right) \nonumber \\
			&=& -\frac{1}{2} (\bm{z} - \bm{\mu})^T \bm{\Lambda} (\bm{z} - \bm{\mu}) + \mathrm{Const.}
		\end{eqnarray}
		$z_1$に依存する項だけを取り出せば
		\begin{eqnarray}
			&& -\frac{1}{2} (\bm{z} - \bm{\mu})^T \bm{\Lambda} (\bm{z} - \bm{\mu}) \nonumber \\
			&=& -\frac{1}{2} \left[ z_1 - \mu_1, z_2 - \mu_2 \right] \left[ \begin{array}{ll} \Lambda_{11} & \Lambda_{12} \\ \Lambda_{21} & \Lambda_{22} \end{array} \right] \left[ \begin{array}{l} z_1 - \mu_1 \\ z_2 - \mu_2 \end{array} \right] \nonumber \\
			&=& -\frac{1}{2} \bigg[ \Big\{ (\Lambda_{11} (z_1 - \mu_1) + \Lambda_{21} (z_2 - \mu_2) \Big\} (z_1 - \mu_1) + \nonumber \\
			&& \qquad \Big\{ \Lambda_{12} (z_1 - \mu_1) + \Lambda_{22} (z_2 - \mu_2) \Big\} (z_2 - \mu_2) \bigg] \nonumber \\
			&=& -\frac{1}{2} \left( \Lambda_{11} (z_1 - \mu_1)^2 + 2 \Lambda_{12} (z_1 - \mu_1)(z_2 - \mu_2) + \right. \nonumber \\
			&& \qquad \left. \Lambda_{22} (z_2 - \mu_2)^2 \right) \quad (\because \Lambda_{21} = \Lambda_{12}) \nonumber \\
			&=& -\frac{1}{2} \Lambda_{11} (z_1 - \mu_1)^2 - \Lambda_{12} (z_1 - \mu_1)(z_2 - \mu_2) + \mathrm{Const.}
		\end{eqnarray}
		これを代入して
		\begin{eqnarray}
			&& \ln \mathcal{N}(\bm{z} | \bm{\mu}, \bm{\Lambda}^{-1}) \nonumber \\
			&=& -\frac{1}{2} \Lambda_{11} (z_1 - \mu_1)^2 - \Lambda_{12} (z_1 - \mu_1)(z_2 - \mu_2) + \mathrm{Const.}
		\end{eqnarray}
		従って
		\begin{eqnarray}
			&& \ln q_1^*(z_1) \nonumber \\
			&=& \mathbb{E}_{z_2} \left[ \ln \mathcal{N}(\bm{z} | \bm{\mu}, \bm{\Lambda}^{-1}) \right] + \mathrm{Const.} \nonumber \\
			&=& \mathbb{E}_{z_2} \left[ -\frac{1}{2} \Lambda_{11} (z_1 - \mu_1)^2 - \Lambda_{12} (z_1 - \mu_1)(z_2 - \mu_2) \right] + \mathrm{Const.} \nonumber \\
			&=& -\frac{1}{2} \Lambda_{11} z_1^2 + \Lambda_{11} \mu_1 z_1 - \Lambda_{12} z_1 \left( \mathbb{E}[z_2] - \mu_2 \right) + \mathrm{Const.}
		\end{eqnarray}
		
		\item これより、$q_1^*(z_1)$は次のように書ける
		\begin{eqnarray}
			&& q_1^*(z_1) \nonumber \\
			&\propto& \exp \left( -\frac{1}{2} \Lambda_{11} z_1^2 + \Lambda_{11} \mu_1 z_1 - \Lambda_{12} z_1 \left( \mathbb{E}[z_2] - \mu_2 \right) \right) \nonumber \\
			&=& \exp \left( -\frac{1}{2} \Lambda_{11} \left( z_1 - (\mu_1 - \Lambda_{11}^{-1} \Lambda_{12} (\mathbb{E}[z_2] - \mu_2)) \right)^2 + \right. \nonumber \\
			&& \qquad \left. \frac{1}{2} \Lambda_{11} \left( \mu_1 - \Lambda_{11}^{-1} \Lambda_{12} (\mathbb{E}[z_2] - \mu_2) \right)^2 \right) \nonumber \\
			&\propto& \exp \left( -\frac{1}{2 \Lambda_{11}^{-1}} \left( z_1 - (\mu_1 - \Lambda_{11}^{-1} \Lambda_{12} (\mathbb{E}[z_2] - \mu_2)) \right)^2 \right) \nonumber \\
			&=& \mathcal{N}(z_1 | m_1, \Lambda_{11}^{-1})
		\end{eqnarray}
		但し$m_1$は次のようにおいた
		\begin{equation}
			m_1 = \mu_1 - \Lambda_{11}^{-1} \Lambda_{12} (\mathbb{E}[z_2] - \mu_2)
		\end{equation}
		
		\item 対称性から、$q_2^*(z_2)$も次のように求められる
		\begin{eqnarray}
			\ln q_2^*(z_2) &=& \mathbb{E}_{z_2}[\ln p(\bm{z})] + \mathrm{Const.} \\
			&=& \mathbb{E}_{z_2}[\ln \mathcal{N}(\bm{z} | \bm{\mu}, \bm{\Lambda}^{-1})] + \mathrm{Const.}
		\end{eqnarray}
		\begin{equation}
			q_2^*(z_2) = \mathcal{N}(z_2 | m_2, \Lambda_{22}^{-1})
		\end{equation}
		但し$m_2$は次のようにおいた
		\begin{equation}
			m_2 = \mu_2 - \Lambda_{22}^{-1} \Lambda_{21} (\mathbb{E}[z_1] - \mu_1)
		\end{equation}
		
		\item これより$q_1^*(z_1), q_2^*(z_2)$は
		\begin{eqnarray}
			q_1^*(z_1) &=& \mathcal{N}(z_1 | m_1, \Lambda_{11}^{-1}) \\
			m_1 &=& \mu_1 - \Lambda_{11}^{-1} \Lambda_{12} (\mathbb{E}[z_2] - \mu_2) \\
			q_2^*(z_2) &=& \mathcal{N}(z_2 | m_2, \Lambda_{22}^{-1}) \\
			m_2 &=& \mu_2 - \Lambda_{22}^{-1} \Lambda_{21} (\mathbb{E}[z_1] - \mu_1)
		\end{eqnarray}
		
		\item $\mathbb{E}[z_2]$は、$z_2$の$q_2(z_2) = \mathcal{N}(z_2 | m_2, \Lambda_{22}^{-1})$による平均であるから、$\mathbb{E}[z_2] = m_2$である(同様に、$\mathbb{E}[z_1] = z_1$)
		\newline
		
		\item これより$m_1, m_2$を連立させれば
		\begin{eqnarray}
			m_1 &=& \mu_1 - \Lambda_{11}^{-1} \Lambda_{12} (m_2 - \mu_2) \\
			m_2 &=& \mu_2 - \Lambda_{22}^{-1} \Lambda_{21} (m_1 - \mu_1)
		\end{eqnarray}
		であるから、$m_1$について解けば
		\begin{eqnarray}
			m_1 &=& \mu_1 - \Lambda_{11}^{-1} \Lambda_{12} (m_2 - \mu_2) \\
			&=& \mu_1 - \Lambda_{11}^{-1} \Lambda_{12} \left( \mu_2 - \Lambda_{22}^{-1} \Lambda_{21} (m_1 - \mu_1) - \mu_2 \right) \\
			&=& \mu_1 + \Lambda_{11}^{-1} \Lambda_{12} \Lambda_{22}^{-1} \Lambda_{21} (m_1 - \mu_1) \\
			&=& \mu_1 + \Lambda_{11}^{-1} \Lambda_{22}^{-1} \Lambda_{12}^2 (m_1 - \mu_1) \\
			&=& \left( 1 - \Lambda_{11}^{-1} \Lambda_{22}^{-1} \Lambda_{12}^2 \right) \mu_1 + \Lambda_{11}^{-1} \Lambda_{22}^{-1} \Lambda_{12}^2 m_1
		\end{eqnarray}
		従って
		\begin{equation}
			\left( 1 - \Lambda_{11}^{-1} \Lambda_{22}^{-1} \Lambda_{12}^2 \right) \mu_1 = \left( 1 - \Lambda_{11}^{-1} \Lambda_{22}^{-1} \Lambda_{12}^2 \right) m_1
		\end{equation}
		精度行列$\bm{\Lambda}$が正則であれば
		\begin{eqnarray}
			&& \Lambda_{11} \Lambda_{12} - \Lambda_{12} \Lambda_{21} \neq 0 \\
			&\Rightarrow& \Lambda_{11} \Lambda_{12} - \Lambda_{12}^2 \neq 0 \\
			&\Rightarrow& \left( \Lambda_{11} \Lambda_{12} \right) \left( 1 - \Lambda_{11}^{-1} \Lambda_{12}^{-1} \Lambda_{12}^2 \right) \neq 0 \\
			&\Rightarrow& \left( 1 - \Lambda_{11}^{-1} \Lambda_{12}^{-1} \Lambda_{12}^2 \right) \neq 0
		\end{eqnarray}
		が成立するから、分布$p(\bm{z})$が非特異(精度行列が正則)ならば
		\begin{equation}
			m_1 = \mu_1
		\end{equation}
		が唯一の解であるほか、対称性から、$m_2$についても以下を得る
		\begin{equation}
			m_2 = \mu_2
		\end{equation}
		
		\item ゆえに、$q_1^*(z_1), q_2^*(z_2)$は
		\begin{eqnarray}
			q_1^*(z_1) &=& \color{red}\mathcal{N}(z_1 | \mu_1, \Lambda_{11}^{-1})\normalcolor \\
			q_2^*(z_2) &=& \color{red}\mathcal{N}(z_2 | \mu_2, \Lambda_{22}^{-1})\normalcolor
		\end{eqnarray}
	\end{itemize} \
	
	\item 注意点1
	\begin{itemize}
		\item $q_1^*(z_1)$は、$q_2^*(z_2)$を使って計算される$p(\bm{z})$の期待値$\mathbb{E}[z_2]$に依存 (逆も成り立つ)
		\item $q_1^*(z_1)$と、$q_2^*(z_2)$は\alert{相互に依存している}ため、2つを同時に求めることはできない
		\item その代わりに、次のように最適化すればよい
		\newline
		\item $q_1(z_1), q_2(z_2)$を適当に初期化したあと、$q_1^*(z_1), q_2^*(z_2)$の式を使って、交互に$q_1(z_1)$と$q_2(z_2)$を更新していく(収束するまでこれを繰り返す)
	\end{itemize} \
	
	\item 注意点2
	\begin{itemize}
		\item $q_1(z_1), q_2(z_2)$の具体的な関数形については、\alert{何の仮定も置かなかった}
		\item $q_i^*(z_i)$がガウス分布だという仮定は置いていないが、$\KL (q || p)$を最適化する\alert{変分推論}によって、\alert{結果的にガウス分布が得られた}
	\end{itemize}
\end{itemize}

\end{frame}

\begin{frame}{変分推論}

\begin{itemize}
	\item $\KL(q || p)$の最適化と$\KL(p || q)$の最適化の比較
	\begin{itemize}
		\item 上記の結果は、$\KL(q || p)$の最適化(エビデンス下界$\mathcal{L}(q)$の最適化)によって得た
		\item $\KL(q || p)$ではなく、\color{red}$\KL(p || q)$を最適化したらどうなるか?\normalcolor
		\item 変分推論ではない、もう一つの近似推論の方法である、\alert{EP法}で使われる考え方
		\newline
		\item $q(\bm{Z})$を$p(\bm{Z} | \bm{X})$に近づけたいのであれば、$\KL(q || p)$と$\KL(p || q)$のどちらを最小化してもよいはず
		\item なぜなら、KLダイバージェンスは、確率分布間の(擬似的な)距離を表すため
	\end{itemize} \
	
	\item $\KL(p || q)$の最適化
	\begin{itemize}
		\item $q(\bm{Z})$が平均場近似によって分解できるとき、$\KL(p || q)$を最適化したい
		\newline
		\item KLダイバージェンス$\KL(p || q)$は、次のように書ける
		\begin{eqnarray}
			\KL(p || q) &\equiv& \KL(p(\bm{Z} | \bm{X}) || q(\bm{Z})) \\
			&=& -\int p(\bm{Z} | \bm{X}) \ln \frac{q(\bm{Z})}{p(\bm{Z} | \bm{X})} d\bm{Z} \\
			&=& -\int p(\bm{Z} | \bm{X}) \left( \ln q(\bm{Z}) - \ln p(\bm{Z} | \bm{X}) \right) d\bm{Z} \\
			&=& -\int p(\bm{Z} | \bm{X}) \ln q(\bm{Z}) d\bm{Z} - \nonumber \\
			&& \qquad -\underbrace{\int p(\bm{Z} | \bm{X}) \ln p(\bm{Z} | \bm{X}) d\bm{Z}}_{\text{$q$には依存しない定数項}} \\
			&=& -\int p(\bm{Z} | \bm{X}) \ln \prod_i q_i(\bm{Z}_i) d\bm{Z} + \mathrm{Const.} \\
			&=& -\int p(\bm{Z} | \bm{X}) \sum_i \ln q_i(\bm{Z}_i) d\bm{Z} + \mathrm{Const.} \\
			&=& -\sum_i \int p(\bm{Z} | \bm{X}) \ln q_i(\bm{Z}_i) d\bm{Z} + \mathrm{Const.}
		\end{eqnarray}
		定数項は、$p(\bm{Z} | \bm{X})$のエントロピーであり、$q$には依存しない
		\newline
		
		\item 各因子$q_j(\bm{Z}_j)$について$\KL(p || q)$を最適化したい
		\item このとき、$i \neq j$となる、全ての$q_i(\bm{Z}_i)$は\alert{固定する}
		\newline
		\item $q_j(\bm{Z}_j)$に依存する項を取り出せば、次のようになる
		\begin{eqnarray}
			&& \sum_i \int p(\bm{Z} | \bm{X}) \ln q_i(\bm{Z}_i) d\bm{Z} \nonumber \\
			&=& \int p(\bm{Z} | \bm{X}) \ln q_j(\bm{Z}_j) d\bm{Z} \\
			&=& \int p(\bm{Z} | \bm{X}) \ln q_j(\bm{Z}_j) d\bm{Z}_1 d\bm{Z}_2 \cdots d\bm{Z}_M \\
			&=& \int p(\bm{Z} | \bm{X}) \ln q_j(\bm{Z}_j) d\bm{Z}_j \left( \prod_{i \neq j} d\bm{Z}_i \right) \\
			&=& \int \left( \int p(\bm{Z} | \bm{X}) \prod_{i \neq j} d\bm{Z}_i \right) \ln q_j(\bm{Z}_j) d\bm{Z}_j
		\end{eqnarray}
		
		\item $q_j(\bm{Z}_j)$は確率分布であるから、以下の条件を満たさなければならない
		\begin{eqnarray}
			&& \int q_j(\bm{Z}_j) d\bm{Z}_j = 1 \quad (\color{red}\text{規格化条件}\normalcolor ) \\
			&& q_j(\bm{Z}_j) \ge 0
		\end{eqnarray}
		
		\item 従って$\KL(p || q)$を$q_j(\bm{Z}_j)$について最適化するとき、\alert{ラグランジュの未定乗数法}を使って、規格化条件を組み込む必要がある
		\newline
		\item $q_j(\bm{Z}_j) \ge 0$という条件は、$\ln q_i(\bm{Z}_i)$という項が既にあるから、何もしなくても\alert{常に満たされる} (ラグランジュ関数に、制約条件を改めて取り入れる必要がない)
		\newline
		
		\item 結局、ラグランジュ汎関数$\mathcal{L}[q_j]$は、次のようになる
		\begin{eqnarray}
			\mathcal{L}[q_j] &=& - \int \left( \int p(\bm{Z} | \bm{X}) \prod_{i \neq j} d\bm{Z}_i \right) \ln q_j(\bm{Z}_j) d\bm{Z}_j + \nonumber \\
			&& \lambda \left( \int q_j(\bm{Z}_j) d\bm{Z}_j - 1 \right)
		\end{eqnarray}
		
		\item 上記は$q_j(\bm{Z}_j)$についての\alert{汎関数}となっていることに注意
		\newline
		
		\item 次の公式を使って、$\mathcal{L}[q_j]$を変分最適化する
		\begin{equation}
			\frac{\delta}{\delta y(x)} \int G(y(x), x) dx = \frac{\partial}{\partial y} G(y(x), x)
		\end{equation}
		
		\item 従って
		\begin{eqnarray}
			&& \frac{\delta}{\delta q_j(\bm{Z}_j)} \mathcal{L}[q_j] \nonumber \\
			&=& - \frac{\delta}{\delta q_j(\bm{Z}_j)} \int \left( \int p(\bm{Z} | \bm{X}) \prod_{i \neq j} d\bm{Z}_i \right) \ln q_j(\bm{Z}_j) d\bm{Z}_j + \nonumber \\
			&& \qquad \frac{\delta}{\delta q_j(\bm{Z}_j)} \lambda \left( \int q_j(\bm{Z}_j) d\bm{Z}_j - 1 \right) \\
			&=& - \frac{\partial}{\partial q_j} \left( \int p(\bm{Z} | \bm{X}) \prod_{i \neq j} d\bm{Z}_i \right) \ln q_j(\bm{Z}_j) + \nonumber \\
			&& \qquad \lambda \frac{\partial}{\partial q_j} q_j(\bm{Z}_j) \\
			&=& - \left( \int p(\bm{Z} | \bm{X}) \prod_{i \neq j} d\bm{Z}_i \right) \frac{1}{q_j(\bm{Z}_j)} + \lambda = 0
		\end{eqnarray}
		
		\item これより、未定乗数$\lambda$は
		\begin{eqnarray}
			&& \lambda q_j(\bm{Z}_j) = \left( \int p(\bm{Z} | \bm{X}) \prod_{i \neq j} d\bm{Z}_i \right) \frac{1}{q_j(\bm{Z}_j)} \\
			&\Rightarrow& \int \lambda q_j(\bm{Z}_j) d\bm{Z}_j = \int \underbrace{\left( \int p(\bm{Z} | \bm{X}) \prod_{i \neq j} d\bm{Z}_i \right)}_{=p(\bm{Z}_j | \bm{X})} d\bm{Z}_j \\
			&\Rightarrow& \lambda \underbrace{\int q_j(\bm{Z}_j) d\bm{Z}_j}_{=1} = \underbrace{\int p(\bm{Z}_j | \bm{X}) d\bm{Z}_j}_{=1} \\
			&\Rightarrow& \lambda = 1
		\end{eqnarray}
		
		\item 結局、最適解$q_j^*(\bm{Z}_j)$は次のようになる
		\begin{eqnarray}
			&& - \left( \int p(\bm{Z} | \bm{X}) \prod_{i \neq j} d\bm{Z}_i \right) \frac{1}{q_j(\bm{Z}_j)} + \lambda = 0 \\
			&\Rightarrow& q_j^*(\bm{Z}_j) = \underbrace{\int p(\bm{Z} | \bm{X}) \prod_{i \neq j} d\bm{Z}_i}_{=p(\bm{Z}_j | \bm{X})} \\
			&\Rightarrow& q_j^*(\bm{Z}_j) = p(\bm{Z}_j | \bm{X})
		\end{eqnarray}
		
		\item $q_j^(\bm{Z}_j)$の最適解は、$p(\bm{Z} | \bm{X})$を、$i \neq j$である全ての$\bm{Z}_i$について周辺化した分布
		\item これは閉じた解であり、繰り返しを必要としない
	\end{itemize} \
	
	\item 最適な因子$q_1(z_1), q_2(z_2)$の計算
	\begin{itemize}
		\item 今回は$p(\bm{z} | \bm{x}) = p(\bm{z})$の場合を考えており、かつ$p(\bm{z}) = \mathcal{N}(\bm{z} | \bm{\mu}, \bm{\Lambda}^{-1})$であった
		\newline
		
		\item 従って、$q_1^*(z_1)$は、$p(\bm{z})$を$z_2$について周辺化すればよいから
		\begin{eqnarray}
			&& q_1^*(z_1) \nonumber \\
			&=& \int p(\bm{z}) dz_2 \nonumber \\
			&=& \int \mathcal{N}(\bm{z} | \bm{\mu}, \bm{\Lambda}^{-1}) dz_2 \\
			&=& \frac{1}{2\pi} \frac{1}{|\bm{\Lambda}|^{-\frac{1}{2}}} \int \exp \left( -\frac{1}{2} (\bm{z} - \bm{\mu})^T \bm{\Lambda} (\bm{z} - \bm{\mu}) \right) dz_2
		\end{eqnarray}
		
		\item ここで、指数の内側を、積分変数$z_2$に依存する項と、そうでない項に分ける
		\begin{eqnarray}
			&& -\frac{1}{2} (\bm{z} - \bm{\mu})^T \bm{\Lambda} (\bm{z} - \bm{\mu}) \nonumber \\
			&=& -\frac{1}{2} \left( \Lambda_{11} (z_1 - \mu_1)^2 + \right. \nonumber \\
			&& \qquad \left. 2 \Lambda_{12} (z_1 - \mu_1)(z_2 - \mu_2) + \Lambda_{22} (z_2 - \mu_2)^2 \right) \\
			&=& -\frac{1}{2} \Lambda_{11} (z_1 - \mu_1)^2 + \Lambda_{12} (z_1 - \mu_1) \mu_2 + \nonumber \\
			&& \qquad -\frac{1}{2} \left( 2 \Lambda_{12} (z_1 - \mu_1) z_2 + \Lambda_{22} (z_2 - \mu_2)^2 \right)
		\end{eqnarray}
		そして
		\begin{eqnarray}
			&& -\frac{1}{2} \left( 2 \Lambda_{12} (z_1 - \mu_1) z_2 + \Lambda_{22} (z_2 - \mu_2)^2 \right) \nonumber \\
			&=& -\frac{1}{2} \left( \Lambda_{22} z_2^2 - 2 \Lambda_{22} \mu_2 z_2 + 2 \Lambda_{12} (z_1 - \mu_1) z_2 + \Lambda_{22} \mu_2^2 \right) \\
			&=& -\frac{1}{2} \left( \Lambda_{22} z_2^2 - 2 \left( \Lambda_{22} \mu_2 - \Lambda_{12} (z_1 - \mu_1) \right) z_2 + \Lambda_{22} \mu_2^2 \right) \\
			&=& -\frac{1}{2} \left( \Lambda_{22} \left( z_2 - \Lambda_{22}^{-1} \left( \Lambda_{22} \mu_2 - \Lambda_{12} (z_1 - \mu_1) \right) \right)^2 - \right. \nonumber \\
			&& \qquad \left. \Lambda_{22} \left( \Lambda_{22}^{-1} \left( \Lambda_{22} \mu_2 - \Lambda_{12} (z_1 - \mu_1) \right) \right)^2 + \Lambda_{22} \mu_2^2 \right) \\
			&=& -\frac{1}{2} \left( \Lambda_{22} \left( z_2 - \Lambda_{22}^{-1} m \right)^2 - \Lambda_{22}^{-1} m^2 + \Lambda_{22} \mu_2^2 \right)
		\end{eqnarray}
		ゆえ($m = \Lambda_{22} \mu_2 - \Lambda_{12} (z_1 - \mu_1)$とおいた)
		\begin{eqnarray}
			&& -\frac{1}{2} (\bm{z} - \bm{\mu})^T \bm{\Lambda} (\bm{z} - \bm{\mu}) \nonumber \\
			&=& -\frac{1}{2} \Lambda_{11} (z_1 - \mu_1)^2 + \Lambda_{12} (z_1 - \mu_1) \mu_2 - \frac{1}{2} \Lambda_{22} \mu_2^2 - \nonumber \\
			&& \qquad \frac{1}{2} \Lambda_{22} \left( z_2 - \Lambda_{22}^{-1} m \right)^2 + \frac{1}{2} \Lambda_{22}^{-1} m^2
		\end{eqnarray}
		
		\item これより、積分変数$z_2$の依存項だけを取り出せたので
		\begin{eqnarray}
			&& \int \exp \left( -\frac{1}{2} (\bm{z} - \bm{\mu})^T \bm{\Lambda} (\bm{z} - \bm{\mu}) \right) dz_2 \nonumber \\
			&=& \exp \left( -\frac{1}{2} \Lambda_{11} (z_1 - \mu_1)^2 + \Lambda_{12} (z_1 - \mu_1) \mu_2 - \right. \nonumber \\
			&& \qquad \left. \frac{1}{2} \Lambda_{22} \mu_2^2 + \frac{1}{2} \Lambda_{22}^{-1} m^2 \right) \nonumber \\
			&& \qquad \int \exp \left( - \frac{1}{2} \Lambda_{22} \left( z_2 - \Lambda_{22}^{-1} m \right)^2 \right) dz_2
		\end{eqnarray}
		であって、右側の積分は、中身が(正規化されていない)ガウス分布であるから
		\begin{eqnarray}
			&& \int \exp \left( - \frac{1}{2} \Lambda_{22} \left( z_2 - \Lambda_{22}^{-1} m \right)^2 \right) dz_2 \nonumber \\
			&=& (2\pi \Lambda_{22}^{-1})^\frac{1}{2} \cdot \int \frac{1}{(2\pi \Lambda_{22}^{-1})^\frac{1}{2}} \exp \left( - \frac{1}{2} \Lambda_{22} \left( z_2 - \Lambda_{22}^{-1} m \right)^2 \right) dz_2 \nonumber \\
			&=& \left( 2\pi \Lambda_{22}^{-1} \right)^\frac{1}{2}
		\end{eqnarray}
		となって、$z_2$を積分により消去できる
		\newline
		
		\item また指数の残りの部分から、$z_1$に依存する項だけを取り出して
		\begin{eqnarray}
			&& -\frac{1}{2} \Lambda_{11} (z_1 - \mu_1)^2 + \Lambda_{12} (z_1 - \mu_1) \mu_2 - \frac{1}{2} \Lambda_{22} \mu_2^2 + \frac{1}{2} \Lambda_{22}^{-1} m^2 \nonumber \\
			&=& -\frac{1}{2} \Lambda_{11} (z_1 - \mu_1)^2 + \Lambda_{12} (z_1 - \mu_1) \mu_2 - \frac{1}{2} \Lambda_{22} \mu_2^2 + \nonumber \\
			&& \qquad \frac{1}{2} \Lambda_{22}^{-1} \left( \Lambda_{22} \mu_2 - \Lambda_{12} (z_1 - \mu_1) \right)^2 \nonumber \\
			&=& -\frac{1}{2} \Lambda_{11} (z_1 - \mu_1)^2 + \Lambda_{12} (z_1 - \mu_1) \mu_2 - \frac{1}{2} \Lambda_{22} \mu_2^2 + \nonumber \\
			&& \qquad \frac{1}{2} \Lambda_{22} \mu_2^2 - \mu_2 \Lambda_{12} (z_1 - \mu_1) + \nonumber \\
			&& \qquad \frac{1}{2} \Lambda_{22}^{-1} \Lambda_{12}^2 (z_1 - \mu_1)^2 \\
			&=& -\frac{1}{2} \left( \Lambda_{11} - \Lambda_{22}^{-1} \Lambda_{12}^2 \right) (z_1 - \mu_1)^2 \\
			&=& -\frac{1}{2} \left( \Lambda_{11} - \Lambda_{22}^{-1} \Lambda_{12}^2 \right) z_1^2 + \nonumber \\
			&& \qquad \frac{1}{2} \left( \Lambda_{11} - \Lambda_{22}^{-1} \Lambda_{12}^2 \right) \mu_1 z_1 + \mathrm{Const.}
		\end{eqnarray}
		
		\item これより結局、$z_2$による積分は次のようになる
		\begin{eqnarray}
			&& \int \exp \left( -\frac{1}{2} (\bm{z} - \bm{\mu})^T \bm{\Lambda} (\bm{z} - \bm{\mu}) \right) dz_2 \nonumber \\
			&=& \left( 2\pi \Lambda_{22}^{-1} \right)^\frac{1}{2} \exp \left( -\frac{1}{2} \left( \Lambda_{11} - \Lambda_{22}^{-1} \Lambda_{12}^2 \right) (z_1 - \mu_1)^2 \right)
		\end{eqnarray}
		
		\item 従って、$q_1^*(z_1)$は次のようになる
		\begin{eqnarray}
			&& q_1^*(z_1) \nonumber \\
			&=& \frac{1}{2\pi} \frac{1}{|\bm{\Lambda}|^{-\frac{1}{2}}} \int \exp \left( -\frac{1}{2} (\bm{z} - \bm{\mu})^T \bm{\Lambda} (\bm{z} - \bm{\mu}) \right) dz_2 \\
			&=& \frac{1}{2\pi} \frac{1}{\left( \Lambda_{11} \Lambda_{22} - \Lambda_{12}^2 \right)^{-\frac{1}{2}}} \left( 2\pi \Lambda_{22}^{-1} \right)^\frac{1}{2} \nonumber \\
			&& \qquad \exp \left( -\frac{1}{2} \left( \Lambda_{11} - \Lambda_{22}^{-1} \Lambda_{12}^2 \right) (z_1 - \mu_1)^2 \right) \\
			&=& \frac{1}{(2\pi)^\frac{1}{2}} \frac{1}{(\Lambda_{11} - \Lambda_{22}^{-1} \Lambda_{12}^2)^{-\frac{1}{2}}} \nonumber \\
			&& \qquad \exp \left( -\frac{1}{2} \left( \Lambda_{11} - \Lambda_{22}^{-1} \Lambda_{12}^2 \right) (z_1 - \mu_1)^2 \right) \\
			&=& \color{red}\mathcal{N}(z_1 | \mu_1, (\Lambda_{11} - \Lambda_{22}^{-1} \Lambda_{12}^2)^{-1})\normalcolor
		\end{eqnarray}
		
		\item 共分散行列$\bm{\Sigma}$を、精度行列$\bm{\Lambda}$を使って次のように定めれば
		\begin{equation}
			\left( \begin{array}{ll} \Sigma_{11} & \Sigma_{12} \\ \Sigma_{21} & \Sigma_{22} \end{array} \right)^{-1} = \left( \begin{array}{ll} \Lambda_{11} & \Lambda_{12} \\ \Lambda_{21} & \Lambda_{22} \end{array} \right) \quad (\Sigma_{12} = \Sigma_{21})
		\end{equation}
		次が成り立つから
		\begin{equation}
			 \left( \begin{array}{ll} \Sigma_{11} & \Sigma_{12} \\ \Sigma_{21} & \Sigma_{22} \end{array} \right) \left( \begin{array}{ll} \Lambda_{11} & \Lambda_{12} \\ \Lambda_{21} & \Lambda_{22} \end{array} \right) = \bm{I}
		\end{equation}
		各成分に注目すれば
		\begin{eqnarray}
			\left\{ \begin{array}{l} \Sigma_{11} \Lambda_{11} + \Sigma_{12} \Lambda_{21} = 1 \\
			\Sigma_{11} \Lambda_{12} + \Sigma_{12} \Lambda_{22} = 0 \end{array} \right.
		\end{eqnarray}
		これを$\Sigma_{11}$について解けば
		\begin{eqnarray}
			&& \Sigma_{11} \Lambda_{11} + \left( -\Lambda_{22}^{-1} \Lambda_{12} \Sigma_{11} \right) \Lambda_{21} = 1 \\
			&\Rightarrow& \Sigma_{11} \left( \Lambda_{11} - \Lambda_{22}^{-1} \Lambda_{12} \Lambda_{21} \right) = 1 \\
			&\Rightarrow& \Sigma_{11} \left( \Lambda_{11} - \Lambda_{22}^{-1} \Lambda_{12}^2 \right) = 1 \\
			&\Rightarrow& \Sigma_{11} = \left( \Lambda_{11} - \Lambda_{22}^{-1} \Lambda_{12}^2 \right)^{-1}
		\end{eqnarray}
		
		\item これから、$q_1^*(z_1)$は次のようにも書ける
		\begin{eqnarray}
			&& q_1^*(z_1) \nonumber \\
			&=& \frac{1}{(2\pi)^\frac{1}{2}} \frac{1}{(\Lambda_{11} - \Lambda_{22}^{-1} \Lambda_{12}^2)^{-\frac{1}{2}}} \nonumber \\
			&& \qquad \exp \left( -\frac{1}{2} \left( \Lambda_{11} - \Lambda_{22}^{-1} \Lambda_{12}^2 \right) (z_1 - \mu_1)^2 \right) \\
			&=& \frac{1}{(2\pi)^\frac{1}{2}} \frac{1}{\Sigma_{11}^\frac{1}{2}} \exp \left( -\frac{1}{2} \Sigma_{11}^{-1} (z_1 - \mu_1)^2 \right) \\
			&=& \color{red}\mathcal{N}(z_1 | \mu_1, \Sigma_{11})\normalcolor
		\end{eqnarray}
		
		\item $q_2^*(z_2)$は、対称性から次のようになる
		\begin{equation}
			q_2^*(z_2) = \color{red}\mathcal{N}(z_2 | \mu_2, (\Lambda_{22} - \Lambda_{11}^{-1} \Lambda_{12}^2)^{-1})\normalcolor = \color{red}\mathcal{N}(z_2 | \mu_2, \Sigma_{22})\normalcolor
		\end{equation}
	\end{itemize} \
	
	\item 2つの解の比較
	\begin{itemize}
		\item $\KL(q || p)$の最小化によって次の解を得た
		\begin{eqnarray}
			q_1^*(z_1) &=& \color{red}\mathcal{N}(z_1 | \mu_1, \Lambda_{11}^{-1})\normalcolor \\
			q_2^*(z_2) &=& \color{red}\mathcal{N}(z_2 | \mu_2, \Lambda_{22}^{-1})\normalcolor
		\end{eqnarray}
		\newline
		
		\item $\KL(p || q)$の最小化では、次の解を得た
		\begin{eqnarray}
			q_1^*(z_1) &=& \color{red}\mathcal{N}(z_1 | \mu_1, \Sigma_{11})\normalcolor \\
			\Sigma_{11} &=& \left( \Lambda_{11} - \Lambda_{22}^{-1} \Lambda_{12}^2 \right)^{-1} \\
			q_2^*(z_2) &=& \color{red}\mathcal{N}(z_2 | \mu_2, \Sigma_{22})\normalcolor \\
			\Sigma_{22} &=& (\Lambda_{22} - \Lambda_{11}^{-1} \Lambda_{12}^2)^{-1}
		\end{eqnarray}
		
		\item $p(\bm{z}) = \mathcal{N}(\bm{z} | \bm{\mu}, \bm{\Sigma})$の平均は$\bm{\mu} = \left[ \mu_1, \mu_2 \right]^T$であったので、いずれの場合も、\alert{平均は正しく捉えている}
		\item しかし、両者の間では、\alert{分散が異なっている}
		\item また、変数$z_1$と$z_2$の間の\alert{相関は消えてなくなっている}
		\newline
		\item 両者の違いを、次の図\ref{fig:comparison-between-kl-divergence}に示す
		\newline
		\item \color{green}緑色\normalcolor の線が真の分布$p(\bm{z})$を表す
		\item 左側の\color{red}赤線\normalcolor は、$\KL(q || p)$の最小化によって得られた分布$q(\bm{z})$
		\item 右側の\color{red}赤線\normalcolor は、$\KL(p || q)$の最小化によって得られた分布$q(\bm{z})$
	\end{itemize}
\end{itemize}

\end{frame}

\begin{frame}{変分推論}

\begin{figure}[h]
	\centering
	\includegraphics[clip,scale=0.75,trim=1cm 14.5cm 1cm 2cm,page=488]{../pattern-recognition-and-machine-learning.pdf}
	\caption{KLダイバージェンスの2つの形の比較}
	\label{fig:comparison-between-kl-divergence}
\end{figure}

\end{frame}

\begin{frame}{変分推論}

\begin{itemize}
	\item 2つの解の比較
	\begin{itemize}
		\item $\KL(q || p)$の最小化で得られる$q(\bm{z})$は、\alert{分散が小さくなる方向に制御されていて}、それと直交する方向の分散は、大きく過小評価されている
		\item 分解による近似では、一般に事後分布$p(\bm{Z} | \bm{X})$を\alert{コンパクトに近似しすぎる}
		\newline
		\item $\KL(p || q)$の最小化で得られる$q(\bm{z})$は、\alert{分散が大きくなる方向に制御されていて}、それと直交する方向の分散は、過大評価されている
		\item 非常に低い確率しか持たないはずの領域にも、多くの確率質量が割り当てられている
	\end{itemize} \
	
	\item 違いが生じる理由
	\begin{itemize}
		\item KLダイバージェンス$\KL(q || p)$は次のようであった
		\begin{equation}
			\KL(q || p) = -\int q(\bm{Z}) \ln \frac{p(\bm{Z} | \bm{X})}{q(\bm{Z})} d\bm{Z}
		\end{equation}
		
		\item $\KL(q || p)$が大きくなる主要因は、$p(\bm{Z})$がほとんど$0$で、$q(\bm{Z})$はそうでない領域
		\item 従って、$\KL(q || p)$を最小化すると、$q(\bm{Z})$は、\color{red}$p(\bm{Z})$が小さい領域を避けるようになる\normalcolor
		\newline
		\item また、$\KL(p || q)$は次のようであった
		\begin{equation}
			\KL(p || q) = -\int p(\bm{Z} | \bm{X}) \ln \frac{q(\bm{Z})}{p(\bm{Z} | \bm{X})} d\bm{Z}
		\end{equation}
		
		\item $\KL(p || q)$が大きくなる主要因は、$q(\bm{Z})$がほとんど$0$で、$p(\bm{Z})$はそうではない領域
		\item 従って、$\KL(p || q)$を最小化すると、$q(\bm{Z})$は、\color{red}$p(\bm{Z})$が$0$でない領域にも、必ず確率を持たせるようになる\normalcolor
		\newline
		\item 別の分布について、この両者の振舞いの違いを観察してみよう
	\end{itemize}
\end{itemize}

\end{frame}

\begin{frame}{変分推論}

\begin{itemize}
	\item 多峰性のある分布を、単峰の分布で近似する場合
	\begin{itemize}
		\item $\KL(q || p)$を最小化する変分近似では、\alert{多数ある峰のうちの1つ}を再現
		\item $\KL(p || q)$を最小化する変分近似では、\alert{全ての峰を平均したような分布}が得られる
		\newline
		\item 多峰性のある分布を平均してしまうと、\alert{予測性能の悪化}をもたらす
		\newline
		\item これらの比較を次の図\ref{fig:comparison-between-kl-divergence-2}に示す
	\end{itemize}
\end{itemize}

\end{frame}

\begin{frame}{変分推論}

\begin{figure}[h]
	\centering
	\includegraphics[clip,scale=0.7,trim=1cm 13cm 1cm 2cm,page=489]{../pattern-recognition-and-machine-learning.pdf}
	\caption{KLダイバージェンスの2つの形の別の比較}
	\label{fig:comparison-between-kl-divergence-2}
\end{figure}

\end{frame}

\begin{frame}{変分推論}

\begin{itemize}
	\item ここまでの話の流れ
	\begin{enumerate}
		\item 分解$q(\bm{Z}) = \prod_i q_i(\bm{Z}_i)$を使ったエビデンス下界$\mathcal{L}(q)$の最適化は、$\KL(q(\bm{Z}) || p(\bm{Z} | \bm{X}))$の最小化と等価である
		\newline
		\item $\KL(q || p)$と、$\KL(p || q)$を最小化する変分近似を、2変数のガウス分布を例として試した
		\newline
		\item $\KL(q || p)$の最小化を使って求めた$q(\bm{Z})$は、事後分布$p(\bm{Z} | \bm{X})$を\alert{コンパクトに近似}する傾向にあった
		\newline
		\item $\KL(p || q)$の最小化によって求めた$q(\bm{Z})$は、事後分布$p(\bm{Z} | \bm{X})$を\alert{大きく捉えて近似}する傾向にあった
	\end{enumerate} \
	
	\item これからの話の流れ
	\begin{itemize}
		\item %TODO
	\end{itemize}
\end{itemize}

\end{frame}

\end{document}
